Ensuring software reliability requires rigorous validation with a sufficient number of test cases, yet test case generation consumes about 15\% of developers’ working time and has long been a bottleneck, increasing maintenance overhead and causing schedule delays \cite{a_survey_on_unit_testing_practices_and_problems}. Traditional remedies—such as random testing and exploratory search—have been explored, but they rarely deliver stable, high code coverage.

The recent advent of large language models (LLMs) has changed the landscape: by supplying only source code or specifications, developers can automatically generate test cases that raise coverage while reducing manual effort, drawing rapid attention from both research and industry \cite{no_more_manual_tests?}. In fact, the 2024 Stack Overflow Developer Survey reports that 46.2\% of respondents are interested in introducing AI tools for test code, and more than 80\% believe such tools will be even more integrated into practice by 2025, underscoring the rising expectations for automated test case generation \cite{ai_in_the_development_workflow_interested,ai_tools_next_year}.

Nevertheless, today’s LLM-generated tests still pose quality challenges \cite{the_good_the_bad}: 

\begin{enumerate}[label=(\roman*)]
    \item Some fail to compile or crash at runtime
\vspace{0.2cm}
    \item Many are redundant yet still fall short on coverage
\vspace{0.2cm}
    \item The stochastic nature of LLMs leads to output variability and poor reproducibility
\end{enumerate}

Previous studies focus on that achieving high coverage requires providing the LLMs with large contextual information. Concrete proposals include method slicing, which decomposes complex functions into condition- or feature-level slices and generates test cases for each, and multi-step reasoning that feeds back coverage metrics to iteratively refine the tests; these approaches improve coverage and quality \cite{hits}.
% concreate: 具体的な

However, these techniques expand the amount of information and the number of tokens turn in the prompt. Then API costs rise nearly linearly with that count. Hence, deploying LLM-based test case generation in real projects demands strategies that controls cost increases caused by prompt bloat while preserving high coverage and quality.

To summarize, the key motivations and challenges addressed in this study are as follows:

\begin{itemize}[label={$\bullet$}]
    \item \textbf{Test case generation remains costly and time-consuming}: consuming 15\% of developers’ working time and contributing to delays and maintenance overhead.
    \vspace{0.2cm}
    
    \item \textbf{Existing methods to get high coverage}: such as method slicing and iterative feedback using coverage metrics—\textbf{require feeding large contextual information} to the LLMs, leading to prompt bloat.
    \vspace{0.2cm}
    
    \item \textbf{Prompt bloat increases tokens}: which in turn causes higher API costs for LLMs, posing a barrier to practical adoption.
\end{itemize}

The main contribution of this paper is to demonstrate that natural language compression and Docstrings optimization can significantly reduce tokens and API costs while preserving 100\% code coverage. These results demonstrate that careful prompt-engineering can deliver predictable, low-risk savings, advancing the deployment of economical LLM-based test case generation.
